% Created 2023-09-15 Fri 11:06
% Intended LaTeX compiler: pdflatex
\documentclass[11pt]{article}
\usepackage[utf8]{inputenc}
\usepackage[T1]{fontenc}
\usepackage{graphicx}
\usepackage{longtable}
\usepackage{wrapfig}
\usepackage{rotating}
\usepackage[normalem]{ulem}
\usepackage{amsmath}
\usepackage{amssymb}
\usepackage{capt-of}
\usepackage{hyperref}
\usepackage[margin=1.1in]{geometry}
\usepackage[margin=1.1in]{geometry}
\usepackage[hang]{footmisc}
\setlength{\parindent}{0pt}
\setlength{\parskip}{1.8ex plus 0.25ex minus 0.25ex}
\setlength{\headsep}{2.5em}
\usepackage{setspace}
\usepackage{hyperref}
\setstretch{1.1}
\usepackage[dvipsnames]{xcolor}
\hypersetup{colorlinks, linkcolor={red!50!black}, citecolor={blue!50!black}, urlcolor={blue!80!black}}
\usepackage{mathpazo}
\usepackage{amssymb}
\author{Nishchay Karle, Obi Obetta, Matt Teichman}
\date{\today}
\title{Attachment Converter Workshop, iPres 2023}
\hypersetup{
 pdfauthor={Nishchay Karle, Obi Obetta, Matt Teichman},
 pdftitle={Attachment Converter Workshop, iPres 2023},
 pdfkeywords={},
 pdfsubject={},
 pdfcreator={Emacs 29.1 (Org mode 9.6.6)}, 
 pdflang={English}}
\begin{document}

\maketitle
Welcome to the Attachment Converter Workshop at iPres 2023!  In this
session, we'll walk you through our new open source application,
Attachment Converter, which batch-converts all attachments in an email
mailbox to preservation formats.

Then next few sections of the handout include background on the
project for your reference, but when the workshop starts, we'll be
working through the material on this handout starting with \hyperref[orged35651]{section}
\ref{orged35651}.

\section{Project Website}
\label{sec:orgbc52658}

The project website is located here:

\url{https://dldc.lib.uchicago.edu/open/attachment-converter}


\section{How to Participate in This Workshop}
\label{sec:orgd4da746}

There are two ways to participate in this workshop.  If you're feeling
tech-savvy, we would encourage you to install the required software in
advance of the workshop and type along with us as we walk through some
illustrative examples of the email format.  If you aren't feeling
tech-savvy, you should be able to just watch and follow along.  Either
way, we are really looking forward to engaging with your questions and
comments as we show you how to use our new tool. If you aren't sure
how tech-savvy you're feeling, the question to ask is whether you're
comfortable opening the Terminal application on your computer and
working at the command prompt.

In the next section, we'll go through how to install the software
you'll need if you want to participate in the workshop by typing along
on your own machines.  If you're planning to simply attend, watch,
listen, and ask questions, please feel free to skip to \hyperref[orged35651]{section}
\ref{orged35651}, which is what we'll be working off of during the
workshop---you won't need to set anything up on your computer in
advance.

\section{Advance Preparation \label{org4ca6623}}
\label{sec:orgaee2422}

If you're planning to type along with us on your computer during the
workshop, then this is the section for you!

The software you'll need to install for the workshop is slightly
different, depending on whether you're working in Windows or macOS.
Either way, you will need to have privileges on your machine that
allow you to install software, so if you're attending this conference
from a work machine, that might be something worth looking into with
your system administrator.

If you're on Windows, please skip to \hyperref[org0b076b3]{section} \ref{org0b076b3}.  If you're on a
Mac, you can proceed to \hyperref[org0cc24f1]{section} \ref{org0cc24f1}.

\subsection{macOS \label{org0cc24f1}}
\label{sec:org932e01b}

If you're on a Mac, you'll need to open a Terminal, then install an
open-source package manager, the \texttt{git} version control system, the
\href{https://www.gnu.org/software/make/}{GNU Make} build tool, and the
\texttt{libpst} package.

Remember: to follow these instructions, you'll need to have the
ability to install software on your machine, so if you don't, you may
want to reach out to your system administrator to see whether they can
grant you the appropriate privileges for doing so.

\subsubsection{Install Homebrew}
\label{sec:orgab08e41}

There are various options for open source package managers on macOS,
but we recommend using \href{https://brew.sh}{Homebrew}.  If you've never used it
before, you'll first need to install XCode Command Line Tools, which
you can do by running this command in your Terminal:

\begin{verbatim}
$ xcode-select --install
\end{verbatim}

Then you can install Homebrew by following the instructions here:

\url{https://brew.sh/}

Or, equivalently, by typing this command:

\footnotesize

\begin{verbatim}
$ /bin/bash -c "$(curl -fsSL https://raw.githubusercontent.com/Homebrew/install/HEAD/install.sh)"
\end{verbatim}

\normalsize

\subsubsection{Install Libpst}
\label{sec:orgbfc95dc}

The last thing we'll ask you to install is Libpst, the software we
will use to convert from Outlook \texttt{.pst} to MBOX format during the
workshop.  To install that, run:

\begin{verbatim}
$ brew install libpst
\end{verbatim}

Once you've reached this point on your Mac, you can skip the next
section---which is our Windows-specific setup instructions---and
proceed straight to section \ref{org54f1244}.

\subsection{Windows (Debian WSL) \label{org0b076b3}}
\label{sec:orgc2d19ca}

Attachment Converter is a UNIX application, which means that in order
to run it on Windows, you'll need to install the \href{https://en.wikipedia.org/wiki/Windows\_Subsystem\_for\_Linux}{Windows Subsystem for
Linux}.  We chose Debian as a Linux distribution for this purpose,
because Debian has full out-of-the-box support for OCaml, the
programming language that Attachment Converter was written in.

So first, you'll install the Debian WSL.  Once that's set up, you'll
open up a Debian WSL Terminal and do everything else from inside that
Terminal, including installing a few more utilities, as well as
running Attachment Converter itself.

Note that you need to have privileges to install software on your
machine to follow these instructions.  If you don't, check with your
system administrator about how to get them.

\subsubsection{Set up the Debian WSL}
\label{sec:orgbc443bc}

To set up the Debian WSL:

\begin{itemize}
\item open up the Microsoft Store application using your Start Menu
\item there should be a search box at the top of the window that opens
\item type ``Debian'' in the search box and hit Enter
\end{itemize}

\begin{itemize}
\item a list of search results will come up; when you find the one
called Debian with an icon that looks like this, click on ``Get''
\item after it's finished installing, to open a Debian WSL terminal, run
``Debian'' form the Start Menu
\item when the terminal first opens up, you will be asked to choose a
username and password for your Linux subsystem
\item when you're typing your password, it won't show anything, but you
will still be typing it
\item don't forget to write those credentials down and keep them available
for reference
\item that will probably be the end of the install process, but if it asks
you to reboot, do that
\end{itemize}

Once you've completed the above steps, if your Debian WSL terminal is
not already open, you can open it by choosing ``Debian'' from the
Windows Start Menu.  If it asks you to log in, use the username and
password you chose during the installation process.

\subsubsection{Prep WSL for installing}
\label{sec:org77ae740}

Before installing everything to your WSL, it will be necessary to
synchronize your machine's installation with the website you're going
to download software from.  To do that, first run this command:

\begin{verbatim}
$ sudo apt update
\end{verbatim}

You should see a bunch of information get printed to the screen about
it connecting to some websites and downloading some information.  It
should also ask you to type the password you chose during the Debian
WSL installation process, since this is the first time you're running
an install command.

Next, the WSL needs the latest version of all the software it came
pre-installed with.  To install all of those software packages in one
go, run this command:

\begin{verbatim}
$ sudo apt upgrade
\end{verbatim}

(Similar to the previous command, but it says \texttt{upgrade} instead of
\texttt{update}.)  As always, you'll see a bunch of information get printed
to the screen.  If it prompts to say yes, say yes.

\subsubsection{Create Installation Directory}
\label{sec:orgd702627}

Next, you need to create the directory the Attachment Converter
program is going to get installed to, which you can do by running
these commands:

\begin{verbatim}
$ cd ~
$ mkdir bin
\end{verbatim}

\subsubsection{Install Version Control Software}
\label{sec:orga40a714}

Now that your UNIX environment is set up, the next step is to install
version control software, which in this case is Git.  To do that, run
this command:

\begin{verbatim}
$ sudo apt install git
\end{verbatim}

This is the utility we will use to get the latest version of the
source code for Attachment Converter, later in these setup
instructions.

If it asks you for your password, use the one that you chose when
installing Debian WSL.  If it asks you to confirm you want to install
Git, say yes.  (You'll be saying yes to everything that comes up in
these instructions)

\subsubsection{Install GNU Make}
\label{sec:orge07f974}

The software we're going to use to compile Attachment Converter is
called Make.  To install it, run this command:

\begin{verbatim}
$ sudo apt install make
\end{verbatim}

\subsubsection{Install Libpst}
\label{sec:org8e45d31}

Finally, we're going to ask you to install Libpst, which is a freely
available utility for converting Outlook \texttt{.pst} files to MBOX
format---the email mailbox format that Attachment Converter uses.  To
install it:

\begin{verbatim}
$ sudo apt install libpst4
\end{verbatim}

Once you've reached this point on your Windows machine, you're ready
to go to \hyperref[org54f1244]{the next section}, in which we show you how to compile
Attachment Converter into an executable that you can run.

\subsection{Compile Attachment Converter \label{org54f1244}}
\label{sec:org378c559}

Now that you're set up with the basic software you need, whether
you're on Windows or a Mac, the next step is to download the source
code for Attachment Converter, compile it into an executable you can
run, and put the executable in a location where your Terminal can see
it.

\subsubsection{Get The Code}
\label{sec:org21f1543}

The first thing we need to do is download the source code for
Attachment Converter.  The simplest way to do that is by using Git.

First, make a new directory to keep your source code in by running
these commands:

\begin{verbatim}
$ cd ~
$ mkdir src
$ cd src
\end{verbatim}

To then download the source code for Attachment Converter using Git,
run this command (you can copy/paste it if it's too long to type):

\begin{verbatim}
$ git clone https://github.com/uchicago-library/attachment-converter.git
\end{verbatim}

That will download all the source code and put it into a directory
called \texttt{attachment-converter} under \texttt{src}.  To go into that directory,
type:

\begin{verbatim}
$ cd attachment-converter
\end{verbatim}

As an aside, if you're on Windows and want to view the contents of a
directory you're in using Windows explorer, you can run this command
to open up an Explorer window in the current directory (note the dot
after the \texttt{explorer.exe} command):

\begin{verbatim}
$ explorer.exe .
\end{verbatim}

If you're on a Mac, you can do the same thing---i.e. view the
directory you're in in Finder using the \texttt{open} command:

\begin{verbatim}
$ open .
\end{verbatim}

Now that you have the source code for Attachment Converter, the next
step is to compile it into an executable program.

\subsubsection{Compiling, the Semi-Automated Way}
\label{sec:org73c0b23}

Let's start with an overview.  The utility we're going to use to
compile Attachment Converter is called Make.  If you're on Windows, we
told you to install that in the previous section.  If you're on a Mac,
then you already have Make installed on your computer.

Attachment Converter has a lot of moving parts, which means that
installing it involves installing some more standard utilities and
copying a lot of different files to a lot of different places in your
home directory.  When you run Make, the full list of things it will do
is:

\begin{itemize}
\item install all the free software that Attachment Converter uses to
convert file attachments
\item install \texttt{opam}, the package manager for the OCaml programming language
\item create a location in your home directory for all \texttt{opam} files to go in
\item install \texttt{dune}, the OCaml build tool, to that location
\item install all third-party OCaml libraries that are necessary to
compile Attachment Converter
\item put a number of different configuration files in places where
Attachment Converter expects them to be, in order to run
\end{itemize}

To compile Attachment Converter and then install it, first make sure
you're in the \\ \texttt{attachment-converter} directory, which is
where Git downloaded and put all of the code:

\begin{verbatim}
$ cd ~/src/attachment-converter
\end{verbatim}

Then from the \texttt{attachment-converter} directory, run:

\begin{verbatim}
$ make home-install
\end{verbatim}

You'll see many messages get printed to the screen, and it should
generally look like it's downloading and installing various programs,
displaying progress bars, and so forth.  This is your cue to go heat
up a pot of tea, because it should take about 5-10 minutes.  The
process may pause at one point to ask you to type in your
administrator password, in which case you should use the one you chose
when you installed Debian.  You may also be prompted to confirm
certain steps with a yes/no prompt; if that happens, just choose ``yes''
each time.  There will also be one or two times when it won't display
anything on the screen, even though it's still working.  You'll know
it's done when you see the final confirmation message.

When the installation process is done, it should print a message that
looks like this:

\begin{verbatim}
Attachment Converter has been installed to ~/bin/attc.
Please ensure that ~/bin is on your path.
\end{verbatim}

Once the installation process is finished, \texttt{\textasciitilde{}/bin} needs to be on your
shell path in order for Attachment Converter to run.  If you don't
know what that means, run this command if you're on Windows:

\begin{verbatim}
$ echo "export PATH=~/bin:$PATH" >> ~/.bashrc
\end{verbatim}

And run this command if you're on a Mac:

\begin{verbatim}
$ echo "export PATH=~/bin:$PATH" >> ~/.zshrc
\end{verbatim}

Then close and reopen your Terminal.

\subsubsection{Compiling, the Manual Step-By-Step Way}
\label{sec:org2ff8ac4}

We've put a lot of work into making the semi-automated installation
process via Make work, but it's complicated and there is always some
chance it will throw an error. If you get an error while running Make,
another thing you can try is to do all the steps that our Make
configuration does indivdually.  Following all these steps should
work, if there's an unexpected error in our Make configuration.
(Though if you do encounter an error, we would love to hear about it,
so that we can fix it and update these instructions!)  Installing
Attachment Converter in that way will probably take you a bit longer.

The full instructions for setting Attachment Converter up in the
non-automated way can be found on our website here:

\url{https://dldc.lib.uchicago.edu/open/attachment-converter/docs/}

That concludes our setup instructions!  The rest of this handout
reflects what we will cover during the workshop proper.

\section{During The Workshop \label{orged35651}}
\label{sec:org6004a2d}

Welcome to our workshop!  We are excited to be here.

Attachment Converter is a command-line utility that batch-converts all
attachments in an email mailbox to preservation formats.  You give it
your email in the form of an MBOX file, and it creates a new MBOX file
with copies of all the attachments in preservation formats, next to
the original attachments in the emails from which they originated.

Let's open the workshop with a quick demo of Attachment Converter.

\subsection{Quick Demo}
\label{sec:org97184e7}

In this demo, we:

\begin{itemize}
\item run Attachment Converter on a small example MBOX containing five
emails
\item the example MBOX contains attachments in the following formats:
\begin{itemize}
\item DOC
\item DOCX
\item JPEG
\item PDF
\end{itemize}
\item those attachments are then converted to, respectively:
\begin{itemize}
\item TXT, PDF-A-1b
\item TXT, PDF-A-1b
\item TIFF
\item PDF-A-1b
\end{itemize}
\end{itemize}

\section{Background}
\label{sec:org58d9970}

You're most likely used to using email clients, whether they're
web-based, like GMail or Hotmail, or run as apps on your computer,
like Thunderbird, Apple Mail, or Outlook.  But what does an email
actually look like, close up?

Interestingly, the email format is not only very old, totally
ubiquitous, and mostly standardized, but it is actually
human-readable!  At the level at which mail servers send and receive
mail, every email is in fact plaintext---that is to say, standard
ASCII characters with no fonts, styling, sizing, or page layout
information in them of the kind you see in word processors.  With most
other software, if we wanted to look at the data it was sending
around, it would be tricky, because it would be raw binary data.  But
with email, the raw data are just sequences of characters you could
read yourself, if you wanted to.

The format of an individual email is pretty standardized, but there
are many different data formats for putting a large collection of
individual emails together into a \emph{mailbox}, such as \textbf{Inbox}, \textbf{Sent
Mail}, or \textbf{Trash}.  Attachment Converter uses one of the oldest and
most universal data formats for mailboxes, called \href{https://www.loc.gov/preservation/digital/formats/fdd/fdd000383.shtml}{MBOX}.

\subsection{The MBOX format}
\label{sec:org50f499b}

Attachment Converter works with mailboxes in MBOX format, which is the
form in which GMail gives you your email when you request a local
backup from your GMail account.

MBOX is an old, standard, and human-readable format.  In other words,
rather than packing large collections of individual emails into a raw
binary data format, the mailbox containing emails is itself also
plaintext.  So in the same way that you can open the full data in an
email up in any text editor, you can open an MBOX up in a text editor
and just look at the information that's in there.

The MBOX format is very simple.  One thing that makes it simple
compared to other formats is that it saves each mailbox in a single
file.  That makes it easy to browse through large sets of mailboxes,
move them around, back them up, and so forth.  Another thing that
makes it simple is that it's nothing other than a format for putting
emails into a sequence.  So an MBOX is essentially a big list: one
email followed by another until you're through all of them.  This is
as opposed to e.g.\@  trying to group/organize the emails in some way, or
trying to include information about what emails are contained in it.
(Later on in this workshop, we'll demonstrate Attachment Converter's
``report'' feature, which you can run on an MBOX when you're browsing
around to get some basic information about it.)

\subsection{The Delimiter: \texttt{From}}
\label{sec:org4c66c54}

Any data arranged into the form of a list on a computer needs some way
of specifying where each item in the list starts and where it
finishes.  Usually, the way we accomplish that is by using a
\emph{delimiter}.  For example, if I were to write down the list ``\texttt{1,2,3}''
as a string of characters, the delimiter in that example would be a
comma and the elements of the list would be \texttt{1}, \texttt{2}, and \texttt{3},
respectively.

In the case of an MBOX, each email begins with a special line of text
that is not considered to be part of the email---only part of the
mailbox.  The rules for constructing a From line go like this:

\begin{itemize}
\item put \texttt{From} at the beginning of the line
\item add one space
\item insert any text you want (i.e. this part is a free text field)
\item end the line
\end{itemize}

So the following are all perfectly good From lines:

\begin{itemize}
\item \texttt{From Matt Teichman}
\item \texttt{From MAILER-DAEMON Fri Jul  8 12:08:34 2011}
\item \texttt{From ...malomadingdong}
\item \texttt{From vd7g8o73 2vfy\textasciicircum{}\&32v///\textbackslash{}\textbackslash{}7y329?\textasciitilde{}````xxx}
\item \texttt{From} (with a space before the line break)
\end{itemize}

One thing that's potentially confusing about From lines is that emails
typically come with a header telling you who the sender was.  That
header usually looks like this:

\begin{verbatim}
From: Bugs Bunny <bugsbunny@uchicago.edu>
\end{verbatim}

A header like that is part of an individual email---not the delimiter
in a mailbox---and a quick way to tell which of these two things
you're looking at is to look for a colon.  So if you see \texttt{From} with
just a space, it's the MBOX delimiter, and if you see \texttt{From:} with a
colon, it's an email header.

So a mailbox in MBOX format is just a sequence of emails with a From
line before each one.  Notionally, it looks like this:

\texttt{From MAILER-DAEMON Fri Jul  8 12:08:34 2011}\\
\textcolor{DarkOrchid}{(first email goes here)}\\
\texttt{From MAILER-DAEMON Fri Jul  8 12:08:34 2011}\\
\textcolor{DarkOrchid}{(second email goes here)}\\
\texttt{From MAILER-DAEMON Fri Jul  8 12:08:34 2011}\\
\textcolor{DarkOrchid}{(third email goes here)}\\
\vspace{-1em}

And so on.  We conclude this section by opening up our example
mailbox.

\subsection{The Anatomy of an Email}
\label{sec:org6cbe00a}

The email specification is very, very complex and has also evolved a
great deal since the technology first emerged in the early 1970s.  It
would take longer than one workshop session to cover all the details,
so what we will instead do is focus on the parts of the email
specification that are most relevant to what Attachment Converter
does.

Fundamentally, an email consists of \emph{headers} followed by a \emph{body}.
The headers are separated from the body by two line breaks.  The
headers function like metadata for an email; they provide an
informational summary about what's in the email either to the
recipient or to the recipient's email software.  The body is the main
part of the email, as in the part the recipient is meant to see.

It might surprise you to hear that you can have an email with no body,
but if you think about it, every time you accidentally hit send before
typing anything could be a case of that, depending on how your email
software decides to do it.  But although you can have emails with no
body, you can't have an email with no headers.  The two required
headers are a date header and a from header, they don't have to come
in any particular order, and they look like this:

\begin{verbatim}
Date: Tue, 10 Aug 2004 14:17:45 -0500
From: Daffy Duck <daffyduck@uchicago.edu>
\end{verbatim}

It's also possible to have a body with just text and no attachment. In
that case, the body of an email looks like what you'd expect:

\begin{verbatim}
Dear Road Runner,

For my whole life, I've wondered what the Warner Brothers foley
artists might have done to create that sound of you blowing a
raspberry with your tongue.  My best guess is that it's the sound of a
person moving their hand over the top of a wet Coke bottle, but I
can't be sure.  If there is anything you could do to clear this
mystery up for me, I would be eternally grateful.

Yours truly,
Porky Pig
\end{verbatim}

In that bare bones type of email, that's it!  When you're finished the
text, you've reached the end of the email.  No attachments yet.  If we
want to start having attachments, we need to look at an extension to
the original email specification that was created in the early 1990s,
called Multipurpose Internet Mail Extensions, or \href{https://datatracker.ietf.org/doc/html/rfc2045}{MIME}.

\subsubsection{MIME types}
\label{sec:org0ffc302}

Normally when we talk about attachments, we think of e.g. a file that
you selected from your computer that you are sending to someone along
with an email.  But an attachment can also just be another email---for
example, that's what a lot of email software does when you forward an
email to someone else: you write what you're going to write, but then
attached to your comment on the forwarded email is the email you are
forwarding.  Believe it or not, in an even more exotic type of case,
you can put a file straight into the body of an email, not as an
attachment.

This is relevant for our purposes, because insofar as the email
specification is concerned, attached emails and attached files enjoy
equal status as attachments, even though colloquially we tend to
assume that an attachment is a file you're loading into an email from
your computer.  We will deal with this potential terminological
confusion by continuing to use the term \emph{attachment} to specifically
mean \emph{file attachment} in contexts where it is clear, but insisting on
the term \emph{file attachment} where there is ambiguity.

Like the email specification, the MIME specification is incredibly
complicated, and allows you to do many, many different things with
emails.  In this workshop, we are going to focus on the fact that MIME
allows you to write an email with multiple parts, each of which are
bodies of an email. So, for example, when you write an email to your
friend with a note in the body, followed by a file you want to send to
them, that is the format in which it will get sent out.

Let's focus on that pretty ordinary type of case. After the two
required headers, the from header and the date header, there will be
the following MIME version header, which is required if you're using
MIME:

\begin{verbatim}
MIME-Version: 1.0
\end{verbatim}

After that, you'll usually find the following optional but
bog-standard content type header, which tells you what kind of
MIME-encoded data you're looking at:

\begin{verbatim}
Content-Type: multipart/mixed; 
          boundary=name-of-boundary-someone-chose
\end{verbatim}

That tells you that what you will be looking at next is a sequence of
MIME-encoded email bodies, and what the delimiter for that sequence of
email bodies will be.  If there are no more headers in the main email,
then the list of headers will end with two line breaks, followed by a
delimiter:

\begin{verbatim}
--name-of-boundary-someone-chose
\end{verbatim}

Note that the delimiter is whatever the application creating the email
chose to name it, under the \texttt{boundary} header field parameter in the
content type header, preceded by two hyphens \texttt{-{}-{}}.  The final
delimiter in the sequence will have two additional hyphens at the end
of it, before the line break.

Back to the first delimiter.  Following the first delimiter will be
the list of MIME headers that provide information about the first
email body in the multipart list. What headers are those?  There could
be a lot of them, but there will normally at least be a content type
header:

\begin{verbatim}
Content-Type: text/plain
\end{verbatim}

After the MIME headers for the first email body in the list are
finished, there will be two line breaks, followed by the message you
wrote to your friend, followed by another boundary:

\begin{verbatim}
Dear Foghorn Leghorn,

Have you ever wondered there is more to life than being a cartoon?
I've wondered about this ever since I first realized I wasn't real
during the short animation, Rabbit Rampage.  Eager to hear your
thoughts.  I'm attaching a book on existential phenomenology by Simone
de Beauvoir that I think might help.

Yours truly,
-Elmer Fudd

--name-of-boundary-someone-chose
\end{verbatim}

The boundary tells us that we're moving onto the second part of the
MIME multipart body, which is a second email body.  This will be the
file attachment, which we look at in the next section.

\subsubsection{Base64 data}
\label{sec:orgb6f578e}

You might be surprised to hear that when you attach a file to an
email, it isn't a file anymore.  That is, it isn't some data sitting
on your hard drive, physically arranged on disk into whatever types of
blocks your operating system understands, the way files on your
computer all are.  The data within a file are nothing more than a
sequence of numbers, and when you move that sequence of numbers off
your hard disk and into an email, the sequence goes straight into the
email.  However, that sequence of numbers doesn't go into the email in
its raw form; if it did, then your text editor would likely get
confused when you tried to open it, and everything would probably look
bizarre.  So under MIME, what we do is convert the data in your file
into a plaintext representation, in which every number in the sequence
is converted to a printable character that you can view in a text
editor.

There are a few ways of doing that conversion, but the one we're going
to focus on is Base64 encoding.  Base64 encoding turns raw binary data
into something that's pretty compact, and also pretty painless to look
at.  Here's what it looks like:

\begin{verbatim}
JVBERi0xLjcKJeLjz9MKMTU1IDAgb2JqCjw8L0ZpbHRlci9GbGF0ZURlY29kZS9MZW5ndGggMTI1
Nz4+c3RyZWFtCnjavVjbjts2EH3XV+gLZN4vwGKBbDYpAjQPbfwDthwDabtpG+T/UQ05JIeySCtp
EOwKulEzZ25nhj68+vL10/U0fx2f3r8e/h3YCH+//zJIM1rnJz2+DFp7vP5r+DD8NrxZVh4+/HP6
PD48HN6/fvc8ssfH8ekZvn86Doe3fORiPF6DDCMm5sbjZXhgTGrGlGNM++U84/Xp8fjH8Oa4KZdv
y7V2YqoWzWRHjGjDU3fhxbOyoGP5X+61WA5Y+3F5dl6uzXJe1vPLHRxyGwdXdvIrKIs6dY6qQa0A
KOkZQDOoVkY4yuM6tdwvl0rjOoDr4jV8rwEmj2thDdwHswyuYWgeL7rB9IwDXaRUlAuuCvrn6JKw
XuH3tv4G3ofnCt1nwVX3YqfasZOsOEx4FOqic8BRwWCWAHV06G0dIe+pki5Q0wbK2f0koxG5m0i2
kUhMTV6t9Bl0AMNMgOOKkbmWbBDnEhltO7pdQ7fnE+e17q4Nvukv41f4Q0ZiBoXguh0+4g1S4sBw
k+BCm5GPjk9aZ40vQaMkdXZF7QkFvkt1s871/K2Lz8M3rL5PMqE2Uy2DnBSVnCGG1Fqy/NyzuEGX
\end{verbatim}

\ldots{}and so on, for pages and pages, depending on how big the file you
Base64-encoded is.  Without going too far into the details of the
encoding, the way it works is that each character in the above
representation stands for a number between 0 and 63.  So you can
represent the data inside of any file as a stream of readable
characters.

This gets us to the next part of our hypothetical email.  After the
end of the first body in the MIME multipart (in this example, the note
from Elmer Fudd), we get some MIME headers that describe the second
part:

\begin{verbatim}
Content-Type: application/pdf
Content-Transfer-Encoding: base64
Content-Disposition: attachment; 
        filename*=utf-8''beauvoir.pdf;
        filename="beauvoir.pdf"
\end{verbatim}

The content type header tells us that what we're about to look at is a
PDF.  That plus the fact that the content disposition says
\texttt{attachment} tells us that it is a file attachment.  The content
disposition header also contains additional header field parameters
called \texttt{filename} and \texttt{filename*}, respectively, which is what your
email software is going to use to determine the name of the file when
the recipient saves it to their machine.  Annoyingly, the \texttt{filename*}
header field parameter is not part of the MIME specification---the
MIME specification just says that your email software can create any
header field parameters it chooses, and also that it can have as many
of them as your email software chooses to create.  But although it is
not actually part of the MIME specification, most email software
you'll use will nonetheless look for that header field parameter and
use it to determine the attachment's file name for when it gets saved
to the recipient's machine.

After the MIME headers for the attached PDF will be the Base64-encoded
data of the PDF itself, which will look like the above text and could
take up many pages, depending on how large the original PDF was.
Although many mail user agents such as GMail impose a 25-megabyte
restriction on attachments, the email specification itself imposes no
such restriction.

After the data for the PDF ends, there will be two line breaks,
followed by one final MIME boundary:

\begin{verbatim}
--name-of-boundary-someone-chose
\end{verbatim}

That concludes our hypothetical example of an email with a prose body
and one file attachment, whose body is a multipart MIME with one
\texttt{text/plain} email body and one PDF file attachment body.

\subsubsection{How To Convert an Outlook \texttt{.pst} to \texttt{.mbox} Format}
\label{sec:orga69355f}

At the University of Chicago, we are usually given mailboxes in one of
two forms for accessioning: MBOX and Outlook PST.  In order to convert
the attachments that occur in our PST collections, we use an
application called \texttt{readpst} to convert them.

Assuming your input \texttt{.pst} file is called \texttt{mailbox.pst} and that you
are currently in the directory where \texttt{mailbox.pst} is located, then
the command to run \texttt{readpst} is as follows:

\begin{verbatim}
$ readpst mailbox.pst
\end{verbatim}

After you run that command, \texttt{readpst} will put a copy of the original
\texttt{.pst} file into the same directory you ran it from, called
\texttt{Inbox.mbox}.  To rename the file to something more reasonable, run
this command:

\begin{verbatim}
$ mv Inbox.mbox mailbox.mbox
\end{verbatim}

Now the original \texttt{.pst} and the converted \texttt{.mbox} have the same
basename.

\subsection{More Detailed Demo}
\label{sec:org4aec76d}

In the next part of the workshop, we'll redo the demo we did at the
beginning, knowing what we now know about how emails are structured
and how MIME multiparts allow us to attach files to them.

We begin by installing Attachment Converter.  For more info on that,
please see our \hyperref[org4ca6623]{pre-workshop installation instructions}.  After the
installation, we'll run Attachment Converter once again on the example
MBOX we used during the initial demo.

Next, we'll take a look at what Attachment Converter did to the
attachments in our example MBOX.

\subsubsection{Looking At The Output}
\label{sec:orgf29dd77}

We saw during the demo earlier that when we opened the converted MBOX
in Apple Mail, the attachments appeared next to the originals.  What
we are now in a position to do is open the converted MBOX in a text
editor and see what it really did, looking at the raw data.

Let's start with the first email, featuring the \texttt{.doc} attachment.  We
know it's going to be a MIME multipart because it features these
headers:

\begin{verbatim}
MIME-Version: 1.0
Content-Type: multipart/mixed; boundary="=-=-="
\end{verbatim}

We can see from this content type header that the boundary for the
multipart will be \texttt{=-=-=}. If we scroll a bit further, we can see the
first part of the MIME multipart is the prose part of the email:

\begin{verbatim}
--=-=-=
Content-Type: text/plain

I really enjoyed looking at that transcription.  Your transcriptionist
is great.
\end{verbatim}

Next is the first file attachment, which is the \texttt{.docx} original.
Here are the headers for it:

\begin{verbatim}
Content-Type: application/msword
Content-Disposition: attachment;
 filename="Episode 146 Gaurav Venkataraman discusses memory in RNA and DNA.doc"
Content-Transfer-Encoding: base64
\end{verbatim}

All good; that's what it looked like before.  Now if we scroll down,
past all the Base64 data for the document, we can see the MIME headers
for the converted PDF-A copy that Attachment Converter created:

\footnotesize

\begin{verbatim}
Content-Type: application/pdf
Content-Disposition: attachment;
        filename="Episode 146 Gaurav Venkataraman discusses memory in RNA and DNA_CONVERTED1694624784.pdf
X-Attachment-Converter: converted;
        source-type=application/msword;
        target-type=application/pdf;
        time-stamp=1694624784;
        conversion-id=soffice-doc-to-pdfa;
        original-file-hash=305817139
\end{verbatim}

\normalsize

We can see here that Attachment Converter added a special header
called \texttt{X-Attachment-Converter} featuring information about the
conversion that took place.  What is says is:

\begin{itemize}
\item the original attachment was a \texttt{.doc} file
\item it got converted to a PDF-A
\item when it was converted (in a special numerical format)
\item the name of this conversion in the configuration file
\item here is number identifying the data from the
original file, for debugging purposes
\end{itemize}

We can tell from the fact that the header \texttt{X-Attachment-Converter}
starts with \texttt{X} that it is a custom header our application put into
the email.  Email software knows it can safely ignore any header that
starts with the letter \texttt{X}, but it can also be set up to pay attention
to the information in there, if it wants to do something special when
it sees a header with a certain name.  In our case, we put this
information there both because Attachment Converter itself needs it,
and because we would like future researchers to have a record of where
converted attachments come from in the email containing them.

Scrolling down to the next attachment in this MIME multipart, we see
the headers for the conversion to plaintext:

\footnotesize

\begin{verbatim}
Content-Transfer-Encoding: base64
Content-Type: text/plain
Content-Disposition: attachment;
        filename="Episode 146 Gaurav Venkataraman discusses memory in RNA and DNA_CONVERTED1694624788.txt
X-Attachment-Converter: converted;
        source-type=application/msword;
        target-type=text/plain;
        time-stamp=1694624788;
        conversion-id=soffice-doc-to-txt;
        original-file-hash=305817139
\end{verbatim}

\normalsize

We can tell we're at the end of the first email in the MBOX, first
because there is a MIME delimiter with an additional two hyphens \texttt{-{}-{}}
after it:

\begin{verbatim}
--=-=-=--
\end{verbatim}

The other reason we know we're at the end of the first email in the
mailbox is that the next line is a From line, i.e.\@  the
beginning of the next email:

\begin{verbatim}
From keith@lib.uchicago.edu Wed Sep 13 16:25:31 2023
\end{verbatim}

Remember, though this From line contains sender and timestamp
information, that is for human eyes only; the only part your email
software pays attention to is the word From, plus the space after, to
tell that this is the beginning of a new email.

\subsection{Advanced Configuration}
\label{sec:orgacbf84c}

Attachment Converter calls out to freely available external utilities
to perform file conversions.  The external utilities it uses by
default are:

\begin{itemize}
\item \texttt{pandoc}
\item \texttt{libreoffice}
\item \texttt{pdftotext}
\item \texttt{vips}
\end{itemize}

The conversions it performs by default are:

\begin{itemize}
\item PDF > plaintext
\item DOC > plaintext
\item DOC > PDF-A-1b
\item DOCX > plaintext
\item DOCX > PDF-A-1b
\item XLS > TSV
\item GIF > TIFF
\item BMP > TIFF
\item JPEG > TIFF
\end{itemize}

However, Attachment Converter also allows you to customize the
conversions it performs with external utilities.  You can set it up to
use the utilities you already have installed to perform new
conversions, you can modify one of the default conversion to use a
different app of your choice, and you can also set it up to use a new
app of your choice to convert a new type of attachment.

Setting Attachment Converter up with a custom configuration requires
more computer expertise than using it with its default, out-of-the-box
configuration.  We won't get into the details of how to set up a new
Attachment Converter configuration from scratch, but we will:

\begin{itemize}
\item customize the configuration in front of you now, so that you can get
an intuitive feel for how much work is involved and what it
generally looks like
\item tell you who to contact at your university for help setting this up,
if you think you might want to do it
\end{itemize}

\subsection{TBD part}
\label{sec:orga360076}

We provided this handout early so that if you wanted to follow the
setup instructions before the workshop, you could.  But we're still
working on the final section on configuration in advance of the
workshop.

During the workshop, you should be able to get a copy of the updated
handout from the Box folder.  If you've reached this point and would
like to use this section on configuration for reference, we would
encourage you to download the latest copy.

If you aren't sure whether the latest materials are up yet, please
check our website for updates:

\url{http://dldc.lib.uchicago.edu/open/attachment-converter}
\end{document}