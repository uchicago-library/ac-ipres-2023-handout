% Created 2023-09-11 Mon 14:33
% Intended LaTeX compiler: pdflatex
\documentclass[11pt]{article}
\usepackage[utf8]{inputenc}
\usepackage[T1]{fontenc}
\usepackage{graphicx}
\usepackage{longtable}
\usepackage{wrapfig}
\usepackage{rotating}
\usepackage[normalem]{ulem}
\usepackage{amsmath}
\usepackage{amssymb}
\usepackage{capt-of}
\usepackage{hyperref}
\usepackage[margin=1.1in]{geometry}
\usepackage[hang]{footmisc}
\setlength{\parindent}{0pt}
\setlength{\parskip}{1.8ex plus 0.25ex minus 0.25ex}
\setlength{\headsep}{2.5em}
\usepackage{setspace}
\usepackage{hyperref}
\setstretch{1.1}
\author{Nishchay Karle, Obi Obetta, Matt Teichman\thanks{teichman@uchicago.edu}}
\date{\today}
\title{Attachment Converter Workshop, iPres 2023}
\hypersetup{
 pdfauthor={Nishchay Karle, Obi Obetta, Matt Teichman},
 pdftitle={Attachment Converter Workshop, iPres 2023},
 pdfkeywords={},
 pdfsubject={},
 pdfcreator={Emacs 29.1 (Org mode 9.6.6)}, 
 pdflang={English}}
\begin{document}

\maketitle
Welcome to the Attachment Converter Workshop at iPres 2023!  In this
session, we'll walk you through our new open source application,
Attachment Converter, which batch-converts all attachments in an email
mailbox to preservation formats.

Then next few sections of the handout include background on the
project for your reference, but when the workshop starts, we'll be
working through the material on this handout starting with section
\ref{org753a434}.

\section{Project Website}
\label{sec:orgd80eb4c}

The project website is located here:

\url{https://dldc.lib.uchicago.edu/open/attachment-converter}


\section{How to Participate in This Workshop}
\label{sec:orgf7902f1}

There are two ways to participate in this workshop.  If you're feeling
tech-savvy, we would encourage you to install the required software in
advance of the workshop and type along with us as we walk through some
illustrative examples of the email format.  If you aren't feeling
tech-savvy, you should be able to just watch and follow along.  Either
way, we are really looking forward to engaging with your questions and
comments as we show you how to use our new tool. If you aren't sure
how tech-savvy you're feeling, the question to ask is whether you're
comfortable opening the Terminal application on your computer and
working at the command prompt.

In the next section, we'll go through how to install the software
you'll need if you want to participate in the workshop by typing along
on your own machines.  If you're planning to simply attend, watch,
listen, and ask questions, please feel free to skip to section
\ref{org753a434}, which is what we'll be working off of during the
workshop---you won't need to set anything up on your computer in
advance.

\section{Advance Preparation \label{orga2f38c7}}
\label{sec:orgdd2a85f}

If you're planning to type along with us on your computer during the
workshop, then this is the section for you!

The software you'll need to install for the workshop is slightly
different, depending on whether you're working in Windows or macOS.
Either way, you will need to have privileges on your machine that
allow you to install software, so if you're attending this conference
from a work machine, that might be something worth looking into with
your system administrator.

If you're on Windows, please skip to section \ref{org09034e1}.  If you're on a
Mac, you can proceed to section \ref{org404995d}.

\subsection{macOS \label{org404995d}}
\label{sec:org46d2026}

If you're on a Mac, you'll need to open a Terminal, then install an
open-source package manager, the \texttt{git} version control system, the \href{https://www.gnu.org/software/make/}{GNU
Make} build tool, and the \texttt{libpst} package.  

Remember: to follow these instructions, you'll need to have the
ability to install software on your machine, so if you don't, you may
want to reach out to your system administrator to see whether they can
grant you the appropriate privileges for doing so.

\subsubsection{Install Homebrew}
\label{sec:orgf329609}

There are various options for open source package managers on macOS,
but we recommend using \href{https://brew.sh}{Homebrew}.  If you've never used it
before, you'll first need to install XCode Command Line Tools, which
you can do by running this command in your Terminal:

\begin{verbatim}
$ xcode-select --install
\end{verbatim}

Then you can install Homebrew by following the instructions here:

\url{https://brew.sh/}

Or, equivalently, by typing this command:

\footnotesize

\begin{verbatim}
$ /bin/bash -c "$(curl -fsSL https://raw.githubusercontent.com/Homebrew/install/HEAD/install.sh)"
\end{verbatim}

\normalsize

\subsubsection{Install Libpst}
\label{sec:org4e25041}

The last thing we'll ask you to install is Libpst, the software we
will use to convert from Outlook \texttt{.pst} to MBOX format during the
workshop.  To install that, run:

\begin{verbatim}
$ brew install libpst
\end{verbatim}

Once you've reached this point on your Mac, you can skip the next
section---which is our Windows-specific setup instructions---and
proceed straight to section \ref{org7b01fff}.

\subsection{Windows (Debian WSL) \label{org09034e1}}
\label{sec:org8565965}

Attachment Converter is a UNIX application, which means that in order
to run it on Windows, you'll need to install the \href{https://en.wikipedia.org/wiki/Windows\_Subsystem\_for\_Linux}{Windows Subsystem for
Linux}.  We chose Debian as a Linux distribution for this purpose,
because Debian has full out-of-the-box support for OCaml, the
programming language that Attachment Converter was written in.

So first, you'll install the Debian WSL.  Once that's set up, you'll
open up a Debian WSL Terminal and do everything else from inside that
Terminal, including installing a few more utilities, as well as
running Attachment Converter itself.

Note that you need to have privileges to install software on your
machine to follow these instructions.  If you don't, check with your
system administrator about how to get them.

\subsubsection{Set up the Debian WSL}
\label{sec:org35b36b5}

To set up the Debian WSL:

\begin{itemize}
\item open up the Microsoft Store application using your Start Menu
\item there should be a search box at the top of the window that opens
\item type ``Debian'' in the search box and hit Enter
\end{itemize}

\begin{itemize}
\item a list of search results will come up; when you find the one
called Debian with an icon that looks like this, click on ``Get''
\item after it's finished installing, to open a Debian WSL terminal, run
``Debian'' form the Start Menu
\item when the terminal first opens up, you will be asked to choose a
username and password for your Linux subsystem
\item when you're typing your password, it won't show anything, but you
will still be typing it
\item don't forget to write those credentials down and keep them available
for reference
\item that will probably be the end of the install process, but if it asks
you to reboot, do that
\end{itemize}

Once you've completed the above steps, if your Debian WSL terminal is
not already open, you can open it by choosing ``Debian'' from the
Windows Start Menu.  If it asks you to log in, use the username and
password you chose during the installation process.

\subsubsection{Prep WSL for installing}
\label{sec:orgaace96f}

Before installing everything to your WSL, it will be necessary to
synchronize your machine's installation with the website you're going
to download software from.  To do that, first run this command:

\begin{verbatim}
$ sudo apt update
\end{verbatim}

You should see a bunch of information get printed to the screen about
it connecting to some websites and downloading some information.  It
should also ask you to type the password you chose during the Debian
WSL installation process, since this is the first time you're running
an install command.

Next, the WSL needs the latest version of all the software it came
pre-installed with.  To install all of those software packages in one
go, run this command:

\begin{verbatim}
$ sudo apt upgrade
\end{verbatim}

(Similar to the previous command, but it says \texttt{upgrade} instead of
\texttt{update}.)  As always, you'll see a bunch of information get printed
to the screen.  If it prompts to say yes, say yes.

\subsubsection{Create Installation Directory}
\label{sec:org3cd909d}

Next, you need to create the directory the Attachment Converter
program is going to get installed to, which you can do by running
these commands:

\begin{verbatim}
$ cd ~
$ mkdir bin
\end{verbatim}

\subsubsection{Install Version Control Software}
\label{sec:orgb4d88c1}

Now that your UNIX environment is set up, the next step is to install
version control software, which in this case is Git.  To do that, run
this command:

\begin{verbatim}
$ sudo apt install git
\end{verbatim}

This is the utility we will use to get the latest version of the
source code for Attachment Converter, later in these setup
instructions.

If it asks you for your password, use the one that you chose when
installing Debian WSL.  If it asks you to confirm you want to install
Git, say yes.  (You'll be saying yes to everything that comes up in
these instructions)

\subsubsection{Install GNU Make}
\label{sec:org91b56e3}

The software we're going to use to compile Attachment Converter is
called Make.  To install it, run this command:

\begin{verbatim}
$ sudo apt install make
\end{verbatim}

\subsubsection{Install Libpst}
\label{sec:orga9d1663}

Finally, we're going to ask you to install Libpst, which is a freely
available utility for converting Outlook \texttt{.pst} files to MBOX
format---the email mailbox format that Attachment Converter uses.  To
install it:

\begin{verbatim}
$ sudo apt install libpst4
\end{verbatim}

Once you've reached this point on your Windows machine, you're ready
to go to the next section, in which we show you how to compile
Attachment Converter into an executable that you can run.

\subsection{Compile Attachment Converter \label{org7b01fff}}
\label{sec:org1d1b548}

Now that you're set up with the basic software you need, whether
you're on Windows or a Mac, the next step is to download the source
code for Attachment Converter, compile it into an executable you can
run, and put the executable in a location where your Terminal can see
it.

\subsubsection{Get The Code}
\label{sec:orgfc70e1a}

The first thing we need to do is download the source code for
Attachment Converter.  The simplest way to do that is by using Git.

First, make a new directory to keep your source code in by running
these commands:

\begin{verbatim}
$ cd ~
$ mkdir src
$ cd src
\end{verbatim}

To then download the source code for Attachment Converter using Git,
run this command (you can copy/paste it if it's too long to type):

\begin{verbatim}
$ git clone https://github.com/uchicago-library/attachment-converter.git
\end{verbatim}

That will download all the source code and put it into a directory
called \texttt{attachment-converter}.  To go into that directory, type:

\begin{verbatim}
$ cd attachment-converter
\end{verbatim}

As an aside, if you're on Windows and want to view the contents of a
directory you're in using Windows explorer, you can run this command
to open up an Explorer window in the current directory (note the dot
after the \texttt{explorer.exe} command):

\begin{verbatim}
$ explorer.exe .
\end{verbatim}

If you're on a Mac, you can do the same thing---i.e. view the
directory you're in in Finder using the \texttt{open} command:

\begin{verbatim}
$ open .
\end{verbatim}

Now that you have the source code for Attachment Converter, the next
step is to compile it into an executable program.

\subsubsection{Compiling, the Semi-Automated Way}
\label{sec:org08c0d89}

\begin{enumerate}
\item Overview
\label{sec:orgc149620}

The utility we're going to use to compile Attachment Converter is
called Make.  If you're on Windows, we told you to install that in the
previous section.  If you're on a Mac, then you already have Make
installed on your computer.

Attachment Converter has a lot of moving parts, which means that
installing it involves installing some more standard utilities and
copying a bunch of different files to a bunch of different places in
your home directory.  When you run Make, the full list of things it
will do is:

\begin{itemize}
\item install all the free software that Attachment Converter uses to
convert file attachments
\item install \texttt{opam}, the package manager for the OCaml programming language
\item create a location in your home directory for all \texttt{opam} files to go in
\item install \texttt{dune}, the OCaml build tool, to that location
\item install all third-party OCaml libraries that are necessary to
compile Attachment Converter
\item put a number of different configuration files in places where
Attachment Converter expects them to be, in order to run
\end{itemize}

\item Instructions
\label{sec:orge19aaf3}

To compile Attachment Converter and then install it, first make sure
you're in the \texttt{attachment-converter} directory, which is where Git
downloaded and put all of the code:

\begin{verbatim}
$ cd ~/src/attachment-converter
\end{verbatim}

Then from the \texttt{attachment-converter} directory, run:

\begin{verbatim}
$ make home-install
\end{verbatim}

You'll see a whole bunch of stuff get printed to the screen, and it it
should generally look like it's downloading and installing various
programs, displaying progress bars, and so forth.  This is your cue to
go heat up a pot of tea, because it should take about 5-10 minutes.
The process may pause at one point to ask you to type in your
administrator password, in which case you should use the one you chose
when you installed Debian.  You may also be prompted to confirm
certain steps with a yes/no prompt; if that happens, just choose ``yes''
each time.  There will also be one or two times when it won't display
anything on the screen, even though it's still working.  You'll know
it's done when you see the final confirmation message.

When the installation process is done, it should print a message that
looks like this:

\begin{verbatim}
Attachment Converter has been installed to ~/bin/attc.
Please ensure that ~/bin is on your path.
\end{verbatim}

Once the installation process is finished, \texttt{\textasciitilde{}/bin} needs to be on your
shell path in order for Attachment Converter to run.  If you don't
know what that means, run this command if you're on Windows:

\begin{verbatim}
$ echo "export PATH=~/bin:$PATH" >> ~/.bashrc
\end{verbatim}

And run this command if you're on a Mac:

\begin{verbatim}
$ echo "export PATH=~/bin:$PATH" >> ~/.zshrc
\end{verbatim}

Then close and reopen your Terminal.
\end{enumerate}

\subsubsection{Compiling, the Manual Step-By-Step Way}
\label{sec:orgeffd146}

We've put a lot of work into making the semi-automated installation
process via Make work, but it's complicated and there is always some
chance it will throw an error. If you get an error while running Make,
another thing you can try is to do all the steps that our Make
configuration does indivdually.  Following all these steps should
work, if there's an unexpected error in our Make configuration.
(Though if you do encounter an error, we would love to hear about it,
so that we can fix it and update these instructions!)  If you install Attachment Converter this way, it will probably 

The full instructions for setting Attachment Converter up in the
non-automated way can be found on our website here:

\url{https://dldc.lib.uchicago.edu/open/attachment-converter/docs/}

That concludes our setup instructions!  The rest of this handout
reflects what we will cover during the workshop proper.

\section{During The Workshop \label{org753a434}}
\label{sec:org14190c6}

Attachment Converter is a command-line utility that batch-converts all
attachments in an email mailbox to preservation formats.  You give it
your email in the form on an MBOX file, and it creates a new MBOX file
with copies of all the attachments in preservation formats, next to
the original attachments in the emails from which they originated.

Let's open the workshop with a quick demo of Attachment Converter.

\subsection{Quick Demo}
\label{sec:org9154436}

In this demo, we:

\begin{itemize}
\item run Attachment Converter on a small example MBOX containing five
emails
\item the example MBOX contains attachments in the following formats:
\begin{itemize}
\item DOC
\item DOCX
\item XLSX
\item JPEG
\item PDF
\end{itemize}
\item those attachments are then converted to, respectively:
\begin{itemize}
\item TXT, PDF-A-1b
\item TXT, PDF-A-1b
\item TSV, PDF-A-1b
\item TIFF
\item PDF-A-1b
\end{itemize}
\end{itemize}

\subsection{Background}
\label{sec:org58d0a81}



\subsubsection{The MBOX format}
\label{sec:orged2aee8}

\subsubsection{The Anatomy of an Email}
\label{sec:orgf0f4c4e}

\begin{enumerate}
\item MIME types
\label{sec:orgaf67918}

\item Base64 data
\label{sec:orgc5336e9}
\end{enumerate}

\subsubsection{How To Convert an Outlook \texttt{.pst} to \texttt{.mbox} Format}
\label{sec:orgff84131}

\subsection{More Detailed Demo}
\label{sec:org880981f}

\subsubsection{Installing Attachment Converter}
\label{sec:orgd5844ef}

\subsubsection{A Simple Example of Running Attachment Converter}
\label{sec:org97cb92d}

\subsubsection{Looking At The Output}
\label{sec:org8d875dd}

\begin{enumerate}
\item The Headers that \texttt{attc} inserts
\label{sec:org17a0327}

\item The Data that \texttt{attc} inserts
\label{sec:orgf52fc4f}
\end{enumerate}

\subsection{Advanced Configuration}
\label{sec:orgac0acaa}

\subsubsection{Attachment Converter's Configuration File}
\label{sec:org51dfcfc}

\subsubsection{A Glance at our Shell Scripts}
\label{sec:org6a09af2}

\begin{enumerate}
\item where they go
\label{sec:org7a86fad}

\item rough overview of what they do
\label{sec:org127ffb1}
\end{enumerate}

\subsubsection{Show and Tell: Here is How To Add a new Utility to Attachment Converter}
\label{sec:orgcdf8a6e}
\end{document}