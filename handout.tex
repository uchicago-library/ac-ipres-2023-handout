% Created 2023-09-12 Tue 11:05
% Intended LaTeX compiler: pdflatex
\documentclass[11pt]{article}
\usepackage[utf8]{inputenc}
\usepackage[T1]{fontenc}
\usepackage{graphicx}
\usepackage{longtable}
\usepackage{wrapfig}
\usepackage{rotating}
\usepackage[normalem]{ulem}
\usepackage{amsmath}
\usepackage{amssymb}
\usepackage{capt-of}
\usepackage{hyperref}
\usepackage[margin=1.1in]{geometry}
\usepackage[margin=1.1in]{geometry}
\usepackage[hang]{footmisc}
\setlength{\parindent}{0pt}
\setlength{\parskip}{1.8ex plus 0.25ex minus 0.25ex}
\setlength{\headsep}{2.5em}
\usepackage{setspace}
\usepackage{hyperref}
\setstretch{1.1}
\usepackage[dvipsnames]{xcolor}
\hypersetup{colorlinks, linkcolor={red!50!black}, citecolor={blue!50!black}, urlcolor={blue!80!black}}
\usepackage{mathpazo}
\author{Nishchay Karle, Obi Obetta, Matt Teichman\thanks{teichman@uchicago.edu}}
\date{\today}
\title{Attachment Converter Workshop, iPres 2023}
\hypersetup{
 pdfauthor={Nishchay Karle, Obi Obetta, Matt Teichman},
 pdftitle={Attachment Converter Workshop, iPres 2023},
 pdfkeywords={},
 pdfsubject={},
 pdfcreator={Emacs 29.1 (Org mode 9.6.6)}, 
 pdflang={English}}
\begin{document}

\maketitle
Welcome to the Attachment Converter Workshop at iPres 2023!  In this
session, we'll walk you through our new open source application,
Attachment Converter, which batch-converts all attachments in an email
mailbox to preservation formats.

Then next few sections of the handout include background on the
project for your reference, but when the workshop starts, we'll be
working through the material on this handout starting with \hyperref[orgdee7dac]{section}
\ref{orgdee7dac}.

\section{Project Website}
\label{sec:org564cc17}

The project website is located here:

\url{https://dldc.lib.uchicago.edu/open/attachment-converter}


\section{How to Participate in This Workshop}
\label{sec:org39c4273}

There are two ways to participate in this workshop.  If you're feeling
tech-savvy, we would encourage you to install the required software in
advance of the workshop and type along with us as we walk through some
illustrative examples of the email format.  If you aren't feeling
tech-savvy, you should be able to just watch and follow along.  Either
way, we are really looking forward to engaging with your questions and
comments as we show you how to use our new tool. If you aren't sure
how tech-savvy you're feeling, the question to ask is whether you're
comfortable opening the Terminal application on your computer and
working at the command prompt.

In the next section, we'll go through how to install the software
you'll need if you want to participate in the workshop by typing along
on your own machines.  If you're planning to simply attend, watch,
listen, and ask questions, please feel free to skip to \hyperref[orgdee7dac]{section}
\ref{orgdee7dac}, which is what we'll be working off of during the
workshop---you won't need to set anything up on your computer in
advance.

\section{Advance Preparation \label{org9e35ac7}}
\label{sec:org7001cf7}

If you're planning to type along with us on your computer during the
workshop, then this is the section for you!

The software you'll need to install for the workshop is slightly
different, depending on whether you're working in Windows or macOS.
Either way, you will need to have privileges on your machine that
allow you to install software, so if you're attending this conference
from a work machine, that might be something worth looking into with
your system administrator.

If you're on Windows, please skip to \hyperref[org390bf19]{section} \ref{org390bf19}.  If you're on a
Mac, you can proceed to \hyperref[orgd9e2771]{section} \ref{orgd9e2771}.

\subsection{macOS \label{orgd9e2771}}
\label{sec:org226999f}

If you're on a Mac, you'll need to open a Terminal, then install an
open-source package manager, the \texttt{git} version control system, the
\href{https://www.gnu.org/software/make/}{GNU Make} build tool, and the
\texttt{libpst} package.

Remember: to follow these instructions, you'll need to have the
ability to install software on your machine, so if you don't, you may
want to reach out to your system administrator to see whether they can
grant you the appropriate privileges for doing so.

\subsubsection{Install Homebrew}
\label{sec:orga8a27ba}

There are various options for open source package managers on macOS,
but we recommend using \href{https://brew.sh}{Homebrew}.  If you've never used it
before, you'll first need to install XCode Command Line Tools, which
you can do by running this command in your Terminal:

\begin{verbatim}
$ xcode-select --install
\end{verbatim}

Then you can install Homebrew by following the instructions here:

\url{https://brew.sh/}

Or, equivalently, by typing this command:

\footnotesize

\begin{verbatim}
$ /bin/bash -c "$(curl -fsSL https://raw.githubusercontent.com/Homebrew/install/HEAD/install.sh)"
\end{verbatim}

\normalsize

\subsubsection{Install Libpst}
\label{sec:org5455be0}

The last thing we'll ask you to install is Libpst, the software we
will use to convert from Outlook \texttt{.pst} to MBOX format during the
workshop.  To install that, run:

\begin{verbatim}
$ brew install libpst
\end{verbatim}

Once you've reached this point on your Mac, you can skip the next
section---which is our Windows-specific setup instructions---and
proceed straight to section \ref{org2bace9e}.

\subsection{Windows (Debian WSL) \label{org390bf19}}
\label{sec:org05c64b0}

Attachment Converter is a UNIX application, which means that in order
to run it on Windows, you'll need to install the \href{https://en.wikipedia.org/wiki/Windows\_Subsystem\_for\_Linux}{Windows Subsystem for
Linux}.  We chose Debian as a Linux distribution for this purpose,
because Debian has full out-of-the-box support for OCaml, the
programming language that Attachment Converter was written in.

So first, you'll install the Debian WSL.  Once that's set up, you'll
open up a Debian WSL Terminal and do everything else from inside that
Terminal, including installing a few more utilities, as well as
running Attachment Converter itself.

Note that you need to have privileges to install software on your
machine to follow these instructions.  If you don't, check with your
system administrator about how to get them.

\subsubsection{Set up the Debian WSL}
\label{sec:org5da744b}

To set up the Debian WSL:

\begin{itemize}
\item open up the Microsoft Store application using your Start Menu
\item there should be a search box at the top of the window that opens
\item type ``Debian'' in the search box and hit Enter
\end{itemize}

\begin{itemize}
\item a list of search results will come up; when you find the one
called Debian with an icon that looks like this, click on ``Get''
\item after it's finished installing, to open a Debian WSL terminal, run
``Debian'' form the Start Menu
\item when the terminal first opens up, you will be asked to choose a
username and password for your Linux subsystem
\item when you're typing your password, it won't show anything, but you
will still be typing it
\item don't forget to write those credentials down and keep them available
for reference
\item that will probably be the end of the install process, but if it asks
you to reboot, do that
\end{itemize}

Once you've completed the above steps, if your Debian WSL terminal is
not already open, you can open it by choosing ``Debian'' from the
Windows Start Menu.  If it asks you to log in, use the username and
password you chose during the installation process.

\subsubsection{Prep WSL for installing}
\label{sec:orgaa2751f}

Before installing everything to your WSL, it will be necessary to
synchronize your machine's installation with the website you're going
to download software from.  To do that, first run this command:

\begin{verbatim}
$ sudo apt update
\end{verbatim}

You should see a bunch of information get printed to the screen about
it connecting to some websites and downloading some information.  It
should also ask you to type the password you chose during the Debian
WSL installation process, since this is the first time you're running
an install command.

Next, the WSL needs the latest version of all the software it came
pre-installed with.  To install all of those software packages in one
go, run this command:

\begin{verbatim}
$ sudo apt upgrade
\end{verbatim}

(Similar to the previous command, but it says \texttt{upgrade} instead of
\texttt{update}.)  As always, you'll see a bunch of information get printed
to the screen.  If it prompts to say yes, say yes.

\subsubsection{Create Installation Directory}
\label{sec:org67c6dda}

Next, you need to create the directory the Attachment Converter
program is going to get installed to, which you can do by running
these commands:

\begin{verbatim}
$ cd ~
$ mkdir bin
\end{verbatim}

\subsubsection{Install Version Control Software}
\label{sec:org71665bf}

Now that your UNIX environment is set up, the next step is to install
version control software, which in this case is Git.  To do that, run
this command:

\begin{verbatim}
$ sudo apt install git
\end{verbatim}

This is the utility we will use to get the latest version of the
source code for Attachment Converter, later in these setup
instructions.

If it asks you for your password, use the one that you chose when
installing Debian WSL.  If it asks you to confirm you want to install
Git, say yes.  (You'll be saying yes to everything that comes up in
these instructions)

\subsubsection{Install GNU Make}
\label{sec:orgca748a0}

The software we're going to use to compile Attachment Converter is
called Make.  To install it, run this command:

\begin{verbatim}
$ sudo apt install make
\end{verbatim}

\subsubsection{Install Libpst}
\label{sec:orgba64547}

Finally, we're going to ask you to install Libpst, which is a freely
available utility for converting Outlook \texttt{.pst} files to MBOX
format---the email mailbox format that Attachment Converter uses.  To
install it:

\begin{verbatim}
$ sudo apt install libpst4
\end{verbatim}

Once you've reached this point on your Windows machine, you're ready
to go to \hyperref[org2bace9e]{the next section}, in which we show you how to compile
Attachment Converter into an executable that you can run.

\subsection{Compile Attachment Converter \label{org2bace9e}}
\label{sec:org26054eb}

Now that you're set up with the basic software you need, whether
you're on Windows or a Mac, the next step is to download the source
code for Attachment Converter, compile it into an executable you can
run, and put the executable in a location where your Terminal can see
it.

\subsubsection{Get The Code}
\label{sec:orgbe682fb}

The first thing we need to do is download the source code for
Attachment Converter.  The simplest way to do that is by using Git.

First, make a new directory to keep your source code in by running
these commands:

\begin{verbatim}
$ cd ~
$ mkdir src
$ cd src
\end{verbatim}

To then download the source code for Attachment Converter using Git,
run this command (you can copy/paste it if it's too long to type):

\begin{verbatim}
$ git clone https://github.com/uchicago-library/attachment-converter.git
\end{verbatim}

That will download all the source code and put it into a directory
called \texttt{attachment-converter} under \texttt{src}.  To go into that directory,
type:

\begin{verbatim}
$ cd attachment-converter
\end{verbatim}

As an aside, if you're on Windows and want to view the contents of a
directory you're in using Windows explorer, you can run this command
to open up an Explorer window in the current directory (note the dot
after the \texttt{explorer.exe} command):

\begin{verbatim}
$ explorer.exe .
\end{verbatim}

If you're on a Mac, you can do the same thing---i.e. view the
directory you're in in Finder using the \texttt{open} command:

\begin{verbatim}
$ open .
\end{verbatim}

Now that you have the source code for Attachment Converter, the next
step is to compile it into an executable program.

\subsubsection{Compiling, the Semi-Automated Way}
\label{sec:orgaf87e6a}

Let's start with an overview.  The utility we're going to use to
compile Attachment Converter is called Make.  If you're on Windows, we
told you to install that in the previous section.  If you're on a Mac,
then you already have Make installed on your computer.

Attachment Converter has a lot of moving parts, which means that
installing it involves installing some more standard utilities and
copying a lot of different files to a lot of different places in your
home directory.  When you run Make, the full list of things it will do
is:

\begin{itemize}
\item install all the free software that Attachment Converter uses to
convert file attachments
\item install \texttt{opam}, the package manager for the OCaml programming language
\item create a location in your home directory for all \texttt{opam} files to go in
\item install \texttt{dune}, the OCaml build tool, to that location
\item install all third-party OCaml libraries that are necessary to
compile Attachment Converter
\item put a number of different configuration files in places where
Attachment Converter expects them to be, in order to run
\end{itemize}

To compile Attachment Converter and then install it, first make sure
you're in the \\ \texttt{attachment-converter} directory, which is
where Git downloaded and put all of the code:

\begin{verbatim}
$ cd ~/src/attachment-converter
\end{verbatim}

Then from the \texttt{attachment-converter} directory, run:

\begin{verbatim}
$ make home-install
\end{verbatim}

You'll see many messages get printed to the screen, and it should
generally look like it's downloading and installing various programs,
displaying progress bars, and so forth.  This is your cue to go heat
up a pot of tea, because it should take about 5-10 minutes.  The
process may pause at one point to ask you to type in your
administrator password, in which case you should use the one you chose
when you installed Debian.  You may also be prompted to confirm
certain steps with a yes/no prompt; if that happens, just choose ``yes''
each time.  There will also be one or two times when it won't display
anything on the screen, even though it's still working.  You'll know
it's done when you see the final confirmation message.

When the installation process is done, it should print a message that
looks like this:

\begin{verbatim}
Attachment Converter has been installed to ~/bin/attc.
Please ensure that ~/bin is on your path.
\end{verbatim}

Once the installation process is finished, \texttt{\textasciitilde{}/bin} needs to be on your
shell path in order for Attachment Converter to run.  If you don't
know what that means, run this command if you're on Windows:

\begin{verbatim}
$ echo "export PATH=~/bin:$PATH" >> ~/.bashrc
\end{verbatim}

And run this command if you're on a Mac:

\begin{verbatim}
$ echo "export PATH=~/bin:$PATH" >> ~/.zshrc
\end{verbatim}

Then close and reopen your Terminal.

\subsubsection{Compiling, the Manual Step-By-Step Way}
\label{sec:orgb806668}

We've put a lot of work into making the semi-automated installation
process via Make work, but it's complicated and there is always some
chance it will throw an error. If you get an error while running Make,
another thing you can try is to do all the steps that our Make
configuration does indivdually.  Following all these steps should
work, if there's an unexpected error in our Make configuration.
(Though if you do encounter an error, we would love to hear about it,
so that we can fix it and update these instructions!)  Installing
Attachment Converter in that way will probably take you a bit longer.

The full instructions for setting Attachment Converter up in the
non-automated way can be found on our website here:

\url{https://dldc.lib.uchicago.edu/open/attachment-converter/docs/}

That concludes our setup instructions!  The rest of this handout
reflects what we will cover during the workshop proper.

\section{During The Workshop \label{orgdee7dac}}
\label{sec:org7ae399f}

Welcome to our workshop!  We are excited to be here.

Attachment Converter is a command-line utility that batch-converts all
attachments in an email mailbox to preservation formats.  You give it
your email in the form on an MBOX file, and it creates a new MBOX file
with copies of all the attachments in preservation formats, next to
the original attachments in the emails from which they originated.

Let's open the workshop with a quick demo of Attachment Converter.

\subsection{Quick Demo}
\label{sec:orgf35cd64}

In this demo, we:

\begin{itemize}
\item run Attachment Converter on a small example MBOX containing five
emails
\item the example MBOX contains attachments in the following formats:
\begin{itemize}
\item DOC
\item DOCX
\item XLSX
\item JPEG
\item PDF
\end{itemize}
\item those attachments are then converted to, respectively:
\begin{itemize}
\item TXT, PDF-A-1b
\item TXT, PDF-A-1b
\item TSV, PDF-A-1b
\item TIFF
\item PDF-A-1b
\end{itemize}
\end{itemize}

\subsection{Background}
\label{sec:orgc08fd92}

You're most likely used to using email clients, whether they're
web-based, like GMail or Hotmail, or run as apps on your computer,
like Thunderbird, Apple Mail, or Outlook.  But what does an email
actually look like, close up?

Interestingly, the email format is not only very old, totally
ubiquitous, and mostly standardized, but it is actually
human-readable!  At the level at which mail servers send and receive
mail, every email is in fact plaintext---that is to say, standard
ASCII characters with no fonts, styling, sizing, or page layout
information in them of the kind you see in word processors.  With most
other software, if we wanted to look at the data it was sending
around, it would be tricky, because it would be raw binary data.  But
with email, the raw data are just sequences of characters you could
read yourself, if you wanted to.

The format in which an individual email is pretty standardized, but
there are many different data formats for putting a large collection
of individual emails together into a \emph{mailbox}, such as \textbf{Inbox}, \textbf{Sent
Mail}, or \textbf{Trash}.  Attachment Converter uses one of the oldest and
most universal data formats for mailboxes, called \href{https://www.loc.gov/preservation/digital/formats/fdd/fdd000383.shtml}{MBOX}.  

\subsubsection{The MBOX format}
\label{sec:org4009708}

MBOX is an old, standard, and human-readable format.  In other words,
rather than packing large collections of individual emails into a raw
binary data format, the mailbox containing emails is itself also
plaintext.  So in the same way that you can open the full data in an
email up in any text editor, you can open an MBOX up in a text editor
and just look at the information that's in there.

The MBOX format is very simple.  One thing that makes it simple
compared to other formats is that it saves each mailbox in a single
file.  That makes it quite easy to browse through large sets of
mailboxes, move them around, back them up, and so forth.  Another
thing that makes it simple is that it's nothing other than a format
for putting emails into a sequence.  So an MBOX is essentially a big
list: one email followed by another until you're through all of them.
This is as opposed to e.g. trying to group/organize the emails in some
way, or trying to include information about what emails are contained
in it.  (Later on in this workshop, we'll demonstrate Attachment
Converter's ``report'' feature, which you can run on an MBOX when you're
browsing around to get some basic information about it.)

\subsubsection{The Delimiter: \texttt{From}}
\label{sec:org5f5bd82}

Any data arranged into the form of a list on a computer needs some way
of specifying where each item in the list starts and where it
finishes.  Usually, the way we accomplish that is by using a
\emph{delimiter}.  For example, if I were to write down the list ``\texttt{1,2,3}''
as a string of characters, the delimiter in that example would be a
comma and the elements of the list would be \texttt{1}, \texttt{2}, and \texttt{3},
respectively.

In the case of an MBOX, each email is separated by a special line of
text that is not considered to be part of the email---only part of the
mailbox.  The rules for constructing a From line go like this:

\begin{itemize}
\item put \texttt{From} at the beginning of the line
\item add one space
\item insert any text you want (i.e. this part is a free text field)
\item end the line
\end{itemize}

So the following are all perfectly good From lines:


\begin{itemize}
\item \texttt{From Matt Teichman}
\item \texttt{From MAILER-DAEMON Fri Jul  8 12:08:34 2011}
\item \texttt{From vd7g8o73 2vfy32v7y3298v7b3r8qv3278r}
\item \texttt{From} (with a space before the line break)
\end{itemize}


One thing that's potentially confusing about From lines is that emails
typically come with a header telling you who the sender was.  That
header usually looks like this:

\begin{verbatim}
From: Bugs Bunny <bugsbunny@uchicago.edu>
\end{verbatim}

A header like that is part of an individual email---not the delimiter
in a mailbox---and a quick way to tell which of these two things
you're looking at is to look for a colon.  So if you see \texttt{From} with
just a space, it's the MBOX delimiter, and if you see \texttt{From:} with a
colon, it's an email header.

So notionally, a mailbox in MBOX format looks like this:

\texttt{From MAILER-DAEMON Fri Jul  8 12:08:34 2011}\\
\textcolor{DarkOrchid}{(first email goes here)}\\
\texttt{From MAILER-DAEMON Fri Jul  8 12:08:34 2011}\\
\textcolor{DarkOrchid}{(second email goes here)}\\
\texttt{From MAILER-DAEMON Fri Jul  8 12:08:34 2011}\\
\textcolor{DarkOrchid}{(third email goes here)}\\
\vspace{-1em}

And so on.  We conclude this section by opening up our example
mailbox.

\subsubsection{The Anatomy of an Email}
\label{sec:org88ada90}

The email specification is very, very complex and has also evolved a
great deal since the technology first emerged in the early 1970s.  It
would take longer than one workshop session to cover all the details,
so what we will instead do is focus on the parts of the email
specification that are most relevant to what Attachment Converter
does.

Fundamentally, an email consists of \emph{headers} followed by a \emph{body}.
The headers function like metadata for an email; they provide an
informational summary about what's in the email either to the
recipient or to the recipient's email software.  The body is the main
part of the email, as in the part the recipient is meant to see.

It might surprise you to hear that you can have an email with no body,
but if you think about it, that happens every time you accidentally
hit send before typing anything.  But although there are emails with
no body, you can't have an email with no headers.  The two required
headers are a date header and a from header, they don't have to come
in any particular order, and they look like this:

\begin{verbatim}
Date: Tue, 10 Aug 2004 14:17:45 -0500
From: Daffy Duck <daffyduck@uchicago.edu>
\end{verbatim}

It's also possible to have:

\begin{itemize}
\item a body with just text and no attachment
\item a body that is nothing but an attachment
\end{itemize}


but you're probably used to emails with multipart bodies, where the
first part is some text and the second part is an attachment

\begin{enumerate}
\item MIME types
\label{sec:orgb148feb}

\item Base64 data
\label{sec:org5dd0b46}
\end{enumerate}

\subsubsection{How To Convert an Outlook \texttt{.pst} to \texttt{.mbox} Format}
\label{sec:org3d69eeb}

\subsection{More Detailed Demo}
\label{sec:orgdff1241}

\subsubsection{Installing Attachment Converter}
\label{sec:orga8cbb7a}

\subsubsection{A Simple Example of Running Attachment Converter}
\label{sec:orgd74692d}

\subsubsection{Looking At The Output}
\label{sec:org81c99e1}

\begin{enumerate}
\item The Headers that \texttt{attc} inserts
\label{sec:org77553e3}

\item The Data that \texttt{attc} inserts
\label{sec:org127cb16}
\end{enumerate}

\subsection{Advanced Configuration}
\label{sec:org80523c1}

\subsubsection{Attachment Converter's Configuration File}
\label{sec:orge2242fa}

\subsubsection{A Glance at our Shell Scripts}
\label{sec:org592e3e8}

\begin{enumerate}
\item where they go
\label{sec:org4dc4250}

\item rough overview of what they do
\label{sec:org43ae65b}
\end{enumerate}

\subsubsection{Show and Tell: Here is How To Add a new Utility to Attachment Converter}
\label{sec:orgf931406}
\end{document}