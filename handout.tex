% Created 2023-08-30 Wed 12:03
% Intended LaTeX compiler: pdflatex
\documentclass[11pt]{article}
\usepackage[utf8]{inputenc}
\usepackage[T1]{fontenc}
\usepackage{graphicx}
\usepackage{longtable}
\usepackage{wrapfig}
\usepackage{rotating}
\usepackage[normalem]{ulem}
\usepackage{amsmath}
\usepackage{amssymb}
\usepackage{capt-of}
\usepackage{hyperref}
\usepackage[margin=1.1in]{geometry}
\usepackage[hang]{footmisc}
\setlength{\parindent}{0pt}
\setlength{\parskip}{1.8ex plus 0.25ex minus 0.25ex}
\setlength{\headsep}{2.5em}
\usepackage{setspace}
\setstretch{1.1}
\author{Nishchay Karle, Obi Obetta, Matt Teichman\thanks{teichman@uchicago.edu}}
\date{\today}
\title{Attachment Converter Workshop, iPres 2023}
\hypersetup{
 pdfauthor={Nishchay Karle, Obi Obetta, Matt Teichman},
 pdftitle={Attachment Converter Workshop, iPres 2023},
 pdfkeywords={},
 pdfsubject={},
 pdfcreator={Emacs 28.2 (Org mode 9.5.5)}, 
 pdflang={English}}
\begin{document}

\maketitle
Welcome to the Attachment Converter Workshop at iPres 2023!  In this
session, we'll walk you through our new open source application,
Attachment Converter, which batch-converts all attachments in an email
mailbox to preservation formats.

There are two ways to participate in this workshop.  If you're feeling
tech-savvy, we would encourage you to install the required software in
advance of the workshop and type along with us as we walk through some
illustrative examples of the email format.  If you aren't feeling
tech-savvy, you should be able to just watch and follow along.  Either
way, we are really looking forward to engaging with your questions and
comments as we show you how to use our new tool.  If you aren't sure
how tech-savvy you're feeling, the question to ask is whether you're
comfortable opening the Terminal application on your computer and
working at the command prompt.

In the next section, we'll go through how to install the software
you'll need if you want to participate in the workshop by typing along
on your own machines.  If you're planning to simply attend, watch,
listen, and ask questions, please feel free to skip that section---you
won't need to set anything up on your computer in advance of the
workshop.

\section*{Advance Preparation}
\label{sec:org6096703}

The software you'll need to install for the workshop is slightly
different, depending on whether you're working in Windows or macOS.
Either way, you will need to have privileges on your machine that
allow you to install software, so if you're attending this conference
from a work machine, that might be something worth checking in
advance.

\subsection*{macOS}
\label{sec:org74d6007}

If you're on a Mac, you'll need to open a Terminal, then install an
open-source package manager, two different kinds of version control
software, and the GNU Make build tool.  We'll go through those steps
next, but if you're on Windows, please skip to the section entitled
\hyperref[sec:org0604b31]{Windows (Debian WSL)}

\subsubsection*{open-source package manager}
\label{sec:org1131f54}

We recommend using Homebrew.  If you've never used it before, you'll
first need to install XCode Command Line Tools, which you can do by
running this command in your Terminal:

\begin{verbatim}
$ xcode-select --install
\end{verbatim}

Then you can install Homebrew by following the instructions here:

\url{https://brew.sh/}

Or, equivalently, by typing this command:

\footnotesize

\begin{verbatim}
$ /bin/bash -c "$(curl -fsSL https://raw.githubusercontent.com/Homebrew/install/HEAD/install.sh)"
\end{verbatim}

\normalsize

\subsection*{Windows (Debian WSL)}
\label{sec:org0604b31}

testing
\end{document}